\documentclass[
	pdftex,
%%%	Zweispaltigkeit, falls gewünscht
%	twocolumn
]{scrartcl}

\usepackage[ngerman]{babel}
\usepackage[utf8]{inputenc}
\usepackage[T1]{fontenc}
\usepackage{tabularx}
\usepackage[juratotoc]{scrjura}
\usepackage{hyperref}
%%%	Einheit, falls benötigt
%\usepackage{units}
%%%%	Geldangaben, falls benötigt
%\usepackage{eurosym}

%%%	Teile sollen arabische (nicht römische) Nummerierung haben.
\renewcommand*{\thepart}{\arabic{part}}

%%%	Erster Absatz soll nicht eingerückt sein.
\setlength{\parindent}{0pt}

%%%	Nummerierung der Sätze in der Vorbemerkung ermöglichen.
\newcounter{PraeSentenceCounter}
\setcounter{PraeSentenceCounter}{0}
\newcommand{\PraeSentence}{\addtocounter{PraeSentenceCounter}{1}\textsuperscript{\arabic{PraeSentenceCounter}}}

\title{Geschäftsordnung der Konferenz Sächsischer Studentenräte}
\subtitle{Geschäftsordnung der Konferenz Sächsischer Studierendenschaften}
\author{LandessprecherInnenrat}
\date{1.\,Januar\,2014}

\begin{document}

\tableofcontents

\maketitle

\begin{contract}

%\minisec{Vorbemerkung}
%\PraeSentence erster Satz der Vorbemerkung: z.B. Die Beitragsordnung (im Folgenden BO genannt) ist Bestandteil der Ordnung der Studentenschaft der Hochschule für Technik und Wirtschaft Dresden (im Folgenden HTW Dresden genannt). 
%\PraeSentence weiterer Satz der Vorbemerkung: Sie wird auf Grund von \S\,29 Abs.\,1 des Gesetzes über die Hochschulen im Freistaat Sachsen (Sächsisches Hochschulgesetz - SächsHSG) vom 10. Dezember 2008 (SächsGVBl. S. 900) vom Studentinnen- und Studentenrat (im Folgenden StuRa genannt) erlassen und bedarf der Genehmigung des Rektorats.

\parnumberfalse\part{Grundsätzliches}\parnumbertrue

\Paragraph{title=Name}

\Sentence Die Konferenz der Sächsischen Studentenräte wird auch Konferenz Sächsischer Studierendenschaften (KSS) genannt.
\Sentence Der Landessprecherrat wird als LandessprecherInnenrat (LSR) bezeichnet.

\Paragraph{title=Rechtsstellung Aufgaben und Mitwirkung}\label{Rechtsstellung Aufgaben und Mitwirkung}

\Sentence Die KSS besteht aus den Studierendenschaften der Hochschulen im Freistaat Sachsen nach \href{http://www.revosax.sachsen.de/Details.do?sid=4771115631165&jlink=p1}{§\,1}\,Abs.\,1\,SächsHSFG.
\Sentence Die Studierendenräte bilden den Zusammenschluss nach \href{http://www.revosax.sachsen.de/Details.do?sid=4771115631165&jlink=p28}{§\,28}\,SächsHSFG.

\Sentence Mitglieder sind alle Studierendenräte der Studierendenschaften nach Abs.\,1.
\Sentence Eine Anerkennung weiterer Mitglieder ist möglich und bedarf einer Ordnung.

\Sentence Die KSS vertritt die Interessen der Studierendenschaften.
\Sentence Sie nimmt die Aufgaben nach \href{http://www.revosax.sachsen.de/Details.do?sid=4771115631165&jlink=p24}{§\,24}\,Abs.\,3, soweit diese einer hochschulübergreifenden Vertretung bedürfen, und \href{http://www.revosax.sachsen.de/Details.do?sid=4771115631165&jlink=p28}{§\,28}\,SächsHSFG wahr.

\Paragraph{title=Wahl des LSR}\label{Wahl des LSR}

Jeder Studierendenrat wählt VertreterInnen in den LSR. Die VertreterInnen werden jährlich gewählt. Das Nähere regelt der jeweilige Studierendenrat.

Einzelne VertreterInnen haben eine Stimme.

VertreterInnen können ihre Stimme an einE StudentIn der KSS gemäß \ref{Rechtsstellung Aufgaben und Mitwirkung} Abs.\,1 Satz\,1 übertragen, soweit der jeweilige Studierendenrat keine abweichende Regelung trifft.

\Paragraph{title=Sitzungen des LSR}\label{Sitzungen des LSR}

Der LSR gibt sich eine Geschäftsordnung.

Insbesondere Mitglieder oder VertreterInnen gemäß \ref{Wahl des LSR} sollen Anträge an den LSR stellen.

Der LSR ist beschlussfähig, wenn die Sitzung ordnungsgemäß einberufen wurde und mehr als die Hälfte
\begin{enumerate}
\item der Stimmen des LSR oder
\item der Studierendenräte mit ihren VertrertInnen
\end{enumerate}
anwesend ist.

Beschlüsse bedürfen mindestens den Erfordernissen nach \href{http://www.revosax.sachsen.de/Details.do?sid=2963315631632&jlink=p54}{§\,54} SächsHSFG.

%\parnumberfalse\part{Name eines weiterer Teils}\parnumbertrue

%Beispiel für eine Einheit \unit[23]{KiB}
%Beispiel für einen Betrag \EUR{23}

%\parnumberfalse\part{Schlussbestimmung}\parnumbertrue

%\Paragraph{title=Schlussbestimmung}

\end{contract}

\newpage

\parnumberfalse\minisec{Beschluss}\parnumbertrue

Diese Fassung der Geschäftsordnung wurde auf der Sitzung des LandessprecherInnenrates am 7.~Dezember~2013 beschlossen.

\end{document}