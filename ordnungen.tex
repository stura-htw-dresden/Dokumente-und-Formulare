\documentclass[
    pdftex,
%    twocolumn
]{scrartcl}

\usepackage[ngerman]{babel}
\usepackage[T1]{fontenc}
\usepackage[utf8]{inputenc}

\usepackage{tabularx}

\usepackage[juratotoc]{scrjura}

\usepackage{hyperref}
\usepackage{units}
\usepackage{eurosym}
\DeclareUnicodeCharacter{20AC}{\euro}	% € as \euro or use \EUR{23,42}

\title{Bezeichnung der Ordnung: z.B. Beitragsordnung}
\newcommand{\Erlassdatum}{11.11.2013}
\subtitle{ausführliche Bezeichnung der Ordnung: z.B. Beitragsordnung mit der Studentinnen- und Studentenschaft der Hochschule für Technik und Wirtschaft Dresden}
\author{StuRa HTW Dresden}
\date{Datum: z.B. \Erlassdatum}

%don't use with scrjura
\usepackage{fancyhdr}
\chead{Bezeichnung der Ordnung: z.B. Beitragsordnung}
\rhead[]{\Erlassdatum}
\lhead[]{}
\pagestyle{fancy}

\setlength{\parindent}{0pt}
%\setlength{\parskip}{\baselineskip}

\renewcommand*{\thepart}{\arabic{part}}

\newcounter{PraeSentenceCounter}
\setcounter{PraeSentenceCounter}{0}
\newcommand{\PraeSentence}{\addtocounter{PraeSentenceCounter}{1}\textsuperscript{\arabic{PraeSentenceCounter}}}

\begin{document}

\tableofcontents

\maketitle

\begin{contract}

\minisec{Vorbemerkung}
\PraeSentence erster Satz der Vorbemerkung: z.\,B. Die Beitragsordnung (im Folgenden BO genannt) ist Bestandteil der Ordnung der Studentenschaft der Hochschule für Technik und Wirtschaft Dresden (im Folgenden HTW Dresden genannt). 
\PraeSentence weiterer Satz der Vorbemerkung: Sie wird auf Grund von \S\,29~Abs.\,1 des Gesetzes über die Hochschulen im Freistaat Sachsen (Sächsisches Hochschulgesetz - SächsHSG) vom 10.~Dezember~2008 (SächsGVBl. S.\,900) vom Studentinnen- und Studentenrat (im Folgenden StuRa genannt) erlassen und bedarf der Genehmigung des Rektorats.

\parnumberfalse\part{Name des ersten Teils}\parnumbertrue
%Paragraph ist für scrjura veraltet, ggf. section ohne scrjura nutzen, Titel dann ohne 'title=' setzen
\Paragraph{title=Name des ersten Paragraphen}
Text eines Paragraphen mit einem Satz.

\Paragraph{title=Name eines weiteren Paragraphen}
\Sentence Text eines ersten Satzes in einem weiteren Paragraphen.
\Sentence Text eines weiteren Satzes in einem weiteren Paragraphen.

\Paragraph{title=Name eines weiteren Paragraphen mit mehreren Absätzen}

Text eines Absatzes mit einem Satz.

\Sentence Text eines ersten Satzes in einem Absatz mit mehreren Sätzen.
\Sentence Text eines weiteren Satzes in einem Absatz mit mehreren Sätzen.

\Paragraph{title=Name eines weiteren Paragraphen mit mehreren Absätzen und mehreren Nummerierungen} \label{Name eines weiteren Paragraphen mit mehreren Absätzen und mehreren Nummerierungen}

Text vor einer Nummerierung
\begin{enumerate}
\item Text der ersten Nummer,
\item Text einer weiteren Nummer;
\end{enumerate}
Text nach einer Nummerierung.

Text des nächsten Absatzes nach einer Nummerierung.

\parnumberfalse\part{Name eines weiterer Teils}\parnumbertrue

Beispiel für eine Einheit \unit[23]{KiB}
Beispiel für einen Betrag \EUR{23}

\parnumberfalse\part{Schlussbestimmung}\parnumbertrue

\Paragraph{title=Schlussbestimmung}

\newpage

\parnumberfalse\minisec{Beschluss}\parnumbertrue

Diese BO wurde auf der Sitzung des StuRa am 11.~Januar~2011 beschlossen.

\end{contract}

\parnumberfalse\minisec{Unterzeichnung des Beschlusses}\parnumbertrue

Dresden, den 11.~Januar~2011

\vfill

\noindent
\begin{tabularx}{\linewidth}{lXlX}
\dotfill				& 	& \dotfill				& \tabularnewline
FAK					&	& RSR					& \tabularnewline
\tiny(Sprecherin StuRa HTW Dresden)	&	& \tiny(Sprecher StuRa HTW Dresden)	& \tabularnewline
\end{tabularx}

\vfill
\vfill

\parnumberfalse\minisec{Genehmigung}\parnumbertrue

Die BO wurde am {Datum: z.B. 11.~Oktober~2011} vom Rektorat genehmigt.

\parnumberfalse\minisec{Unterzeichnung der Genehmigung}\parnumbertrue

Dresden, den {Datum: z.B. 11.~Oktober~2011}

\vfill

\noindent
\begin{tabularx}{\linewidth}{lXlX}
\dotfill			&	& 	& \tabularnewline
Prof.\,Dr.\,Ing. Roland Stenzel	&	& 	& \tabularnewline
\tiny(Rektorat HTW Dresden)	&	& 	& \tabularnewline
\end{tabularx}

\vfill
\vfill

\end{document}