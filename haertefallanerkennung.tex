%%%%%%%%%%%%%%%%%%%%%%%%%%%%%%%%%%%%%%%%%%%%%%%%%%%%%%%%%%%%%%%%%%%%%%%%%%%%%%%%
%% filename:	Haertefallanerkennung.tex
%% template:	Mon, 07 May 2012 13:01:14 +0200 (Hermann Lorenz)
%% author:		Fabian Kunde
%% Übersetzen:
%%	`mkdir -p aux`
%%	`lualatex -output-directory aux Haertefallanerkennung.tex`
%%	Dazu muss die Datei 'stura_logo.pdf' im Verzeichnis liegen.
%%	Es ist das Datum anzupassen, damit auf dem ausgedruckten Formular sofort
%%	ersichtlich ist, wann die letzte Änderung vorgenommen wurde und ob
%%	das Formular auf dem aktuellen Stand ist. ('\ofoot{Stand: ...}`)
%%%%%%%%%%%%%%%%%%%%%%%%%%%%%%%%%%%%%%%%%%%%%%%%%%%%%%%%%%%%%%%%%%%%%%%%%%%%%%%%
%%%%%%%%%%%%%%%%%%%%%%%%%%%%%%%%%%%%%%%%%%%%%%%%%%%%%%%%%%%%%%%%%%%%%%%%%%%%%%%%
\documentclass[%
	ngerman,	% an alle Packete weitergeben
	parskip=half,
	% für Package typearea
	paper=a4,% Druckbereich A4
	pagesize=auto	% Druckbereich als Papiergröße über-
			% nehmen
	]{scrartcl}


%%%%%%%%%%%%%%%%%%%%%%%%%%%%%%%%%%%%%%%%%%%%%%%%%%%%%%%%%%%%%%%%%%%%%%%%%%%%%%%%
%% LuaLaTeX gesondert behandeln %%%%%%%%%%%%%%%%%%%%%%%%%%%%%%%%%%%%%%%%%%%%%%%%
%%%%%%%%%%%%%%%%%%%%%%%%%%%%%%%%%%%%%%%%%%%%%%%%%%%%%%%%%%%%%%%%%%%%%%%%%%%%%%%%
\usepackage{ifluatex}
\usepackage{eurosym}

\usepackage{geometry}
\usepackage{scrpage2}
\geometry{left=1cm,right=1cm,top=1cm,bottom=1.5cm}

%%%%%%%%%%%%%%%%%%%%%%%%%%%%%%%%%%%%%%%%%%%%%%%%%%%%%%%%%%%%%%%%%%%%%%%%%%%%%%%%
%% Lokalisierung %%%%%%%%%%%%%%%%%%%%%%%%%%%%%%%%%%%%%%%%%%%%%%%%%%%%%%%%%%%%%%%
%%%%%%%%%%%%%%%%%%%%%%%%%%%%%%%%%%%%%%%%%%%%%%%%%%%%%%%%%%%%%%%%%%%%%%%%%%%%%%%%
\ifluatex
	\usepackage{fontspec}
\else
	\usepackage[utf8]{inputenx}	% Umlaute direkt eingeben
	\usepackage[T1]{fontenc}	% Wörter mit Umlaute umbrechen
\fi

\usepackage{babel}				% deutsche Bezeichner
\usepackage[autostyle,german=guillemets]{csquotes}	% \enquote{}

%%%%%%%%%%%%%%%%%%%%%%%%%%%%%%%%%%%%%%%%%%%%%%%%%%%%%%%%%%%%%%%%%%%%%%%%%%%%%%%%
%% Font %%%%%%%%%%%%%%%%%%%%%%%%%%%%%%%%%%%%%%%%%%%%%%%%%%%%%%%%%%%%%%%%%%%%%%%%
%%%%%%%%%%%%%%%%%%%%%%%%%%%%%%%%%%%%%%%%%%%%%%%%%%%%%%%%%%%%%%%%%%%%%%%%%%%%%%%%
\ifluatex
	\setmainfont[Mapping=tex-text]{Linux Libertine O}
	\setsansfont[Mapping=tex-text]{Linux Biolinum O}
	\setmonofont[Mapping=tex-text,Scale=0.85]{DejaVu Sans Mono}
\else
	\usepackage{libertine}
	\usepackage[scaled=0.9]{DejaVuSansMono}
\fi

%%%%%%%%%%%%%%%%%%%%%%%%%%%%%%%%%%%%%%%%%%%%%%%%%%%%%%%%%%%%%%%%%%%%%%%%%%%%%%%%
%% Tabellen %%%%%%%%%%%%%%%%%%%%%%%%%%%%%%%%%%%%%%%%%%%%%%%%%%%%%%%%%%%%%%%%%%%%
%%%%%%%%%%%%%%%%%%%%%%%%%%%%%%%%%%%%%%%%%%%%%%%%%%%%%%%%%%%%%%%%%%%%%%%%%%%%%%%%
\usepackage{tabularx}
\usepackage{booktabs}	% \toprule\midrule\bottomrule
			% \addlinespace

%%%%%%%%%%%%%%%%%%%%%%%%%%%%%%%%%%%%%%%%%%%%%%%%%%%%%%%%%%%%%%%%%%%%%%%%%%%%%%%%
%% Bilder %%%%%%%%%%%%%%%%%%%%%%%%%%%%%%%%%%%%%%%%%%%%%%%%%%%%%%%%%%%%%%%%%%%%%%
%%%%%%%%%%%%%%%%%%%%%%%%%%%%%%%%%%%%%%%%%%%%%%%%%%%%%%%%%%%%%%%%%%%%%%%%%%%%%%%%
\usepackage{graphicx}	% \includegraphics{bild.pdf}

%%%%%%%%%%%%%%%%%%%%%%%%%%%%%%%%%%%%%%%%%%%%%%%%%%%%%%%%%%%%%%%%%%%%%%%%%%%%%%%%
%% Mathematische Symbole %%%%%%%%%%%%%%%%%%%%%%%%%%%%%%%%%%%%%%%%%%%%%%%%%%%%%%%
%%%%%%%%%%%%%%%%%%%%%%%%%%%%%%%%%%%%%%%%%%%%%%%%%%%%%%%%%%%%%%%%%%%%%%%%%%%%%%%%
\usepackage{amssymb}
\usepackage{amsmath}
\ifluatex
	\usepackage{unicode-math}
	\setmathfont{Asana-Math.otf}
	%\setmathfont{lmmath-regular.otf}
	%\setmathfont{xits-math.otf}
\else
	\usepackage{amssymb}
	\usepackage{amsfonts}
\fi
\usepackage{nicefrac}

%%%%%%%%%%%%%%%%%%%%%%%%%%%%%%%%%%%%%%%%%%%%%%%%%%%%%%%%%%%%%%%%%%%%%%%%%%%%%%%%
%% Sonstige Symbole %%%%%%%%%%%%%%%%%%%%%%%%%%%%%%%%%%%%%%%%%%%%%%%%%%%%%%%%%%%%
%%%%%%%%%%%%%%%%%%%%%%%%%%%%%%%%%%%%%%%%%%%%%%%%%%%%%%%%%%%%%%%%%%%%%%%%%%%%%%%%
\usepackage{xspace}
\usepackage{paralist}
\usepackage{xcolor}

%%%%%%%%%%%%%%%%%%%%%%%%%%%%%%%%%%%%%%%%%%%%%%%%%%%%%%%%%%%%%%%%%%%%%%%%%%%%%%%%
%% pdf-links %%%%%%%%%%%%%%%%%%%%%%%%%%%%%%%%%%%%%%%%%%%%%%%%%%%%%%%%%%%%%%%%%%%
%%%%%%%%%%%%%%%%%%%%%%%%%%%%%%%%%%%%%%%%%%%%%%%%%%%%%%%%%%%%%%%%%%%%%%%%%%%%%%%%
\usepackage{hyperref}
\definecolor{linkcolor}{rgb}{.2,.2,.7}
\hypersetup{
	colorlinks,	% farbige Links, nicht umrandet
	linkcolor=linkcolor,	% Farbe fuer dokumentinterne Links
	urlcolor=linkcolor,	% Farbe fuer dokumentexterne Links
	linktoc=all,	% in Verzeichnissen Zahlen und Texte verlinken
	unicode,	% Umlaute in PDF-Strings
}

%%%%%%%%%%%%%%%%%%%%%%%%%%%%%%%%%%%%%%%%%%%%%%%%%%%%%%%%%%%%%%%%%%%%%%%%%%%%%%%%
%% eigene Macros %%%%%%%%%%%%%%%%%%%%%%%%%%%%%%%%%%%%%%%%%%%%%%%%%%%%%%%%%%%%%%%
%%%%%%%%%%%%%%%%%%%%%%%%%%%%%%%%%%%%%%%%%%%%%%%%%%%%%%%%%%%%%%%%%%%%%%%%%%%%%%%%
\newcommand{\nachweis}[1]{\emph{(#1)}}

\newsavebox{\InfoBlockBox}
\newenvironment{InfoBlock}[1]{%
%
	\par\vspace{1em}\noindent\textbf{\textsf{\large#1}}\par\noindent%
		\begin{lrbox}{\InfoBlockBox}%
		\begin{minipage}{\dimexpr\linewidth-2\fboxrule-2\fboxsep}%
	}{%
		\end{minipage}%
		\end{lrbox}
		\fbox{\usebox{\InfoBlockBox}}%
	}

%%==============================================================================
%% Allgemeine Formatierungen/Einstellungen
%%==============================================================================
\definecolor{bg-gray}{gray}{.9}
\linespread{1.25}	% 1.5-facher Zeilenabstand (1.2 * 1.25 = 1.5)

\ofoot{Stand: 12.\,Februar~2014\\[1cm]}
\cfoot{\thepage\\[1cm]}
\pagestyle{scrheadings}


%%==============================================================================
%% Formatierungen der Formularfelder
%%==============================================================================
\def\DefaultHeightofCheckBox{8px}
\def\DefaultWidthofCheckBox{8px}
\def\DefaultHeightofTextField{12px}

\def\LayoutCheckField#1#2{\mbox{\mbox{#2}~#1}\xspace}
\def\LayoutTextField#1#2{\mbox{\mbox{#1}~~\mbox{#2}}\xspace}


%%==============================================================================
%% Formatierungsbefehle
%%==============================================================================
\newcommand{\feldumbruch}{\\[1ex]}
\newcommand{\feldpar}{\vspace{\dimexpr1ex+2ex}}
\newcommand{\halflinefield}[2][]{%
	\partlinefield[#1]{.32}{.15}{#2}%
	}
\newcommand{\partlinefield}[4][]{%
	\makebox[\dimexpr#2\linewidth+#3\linewidth+1em][l]{%
		\TextField[width=#2\linewidth,#1]
			{\makebox[#3\linewidth][r]{#4}}%
	}%
}

%%==============================================================================
%% PDF-Metadaten
%%==============================================================================
\hypersetup{
	pdftitle={Formular Antrag auf Haertefallanerkennung},
	pdfauthor={StuRa HTW Dresden}%,
}

\newcommand{\Logo}{%
	./img/stura_logo.pdf%
}

\begin{document}
\begin{Form}

\parbox{\dimexpr\textwidth-3cm}{%
	\noindent%
	\textbf{\huge Antrag auf Härtefallanerkennung}%
}%
%%------------------------------------------------------------------------------
%% Logo
%%------------------------------------------------------------------------------
%\hfill%
\parbox{3cm}{%
%	\IfFileExists{\Logo}{	\includegraphics[width=3cm]{\Logo}	}{	Die Datei "\Logo" fehlt!	}%
}

%%------------------------------------------------------------------------------
%% Allgemeine Angaben
%%------------------------------------------------------------------------------
\vspace{-8.0mm}
\begin{InfoBlock}{Allgemeine Angaben}
	\partlinefield[name=nachname]{.32}{.15}{Name}%
	\partlinefield[name=vorname]{.37}{.10}{Vorname}%
	\feldumbruch
	\partlinefield[name=snummer]{.15}{.15}{Bibl.\,-\,Nr.}

	\feldpar

	\partlinefield[name=strasse]{.5}{.15}{Straße}\feldumbruch
	\partlinefield[name=plz]{.1}{.15}{PLZ, Ort},%
		\TextField[width=.5\linewidth,name=ort]{~}

	\feldpar

	\partlinefield[name=telefon]{.32}{.15}{Telefonnummer}%
	\partlinefield[name=email]{.37}{.10}{E-Mail}%
\end{InfoBlock}
%%------------------------------------------------------------------------------
%% Bankverbindung
%%------------------------------------------------------------------------------

\vspace{-3mm}
\begin{InfoBlock}{Bankverbindung}
	\partlinefield[name=iban]{.37}{.15}{IBAN}%
	\partlinefield[name=bic]{.32}{.10}{BIC}%
	\feldumbruch
	\halflinefield[name=kontoinhaber]{Kontoinhaber}%
	\halflinefield[name=kreditinstitut]{Kreditinstitut}%
\end{InfoBlock}


%%------------------------------------------------------------------------------
%% Unterschrift
%%------------------------------------------------------------------------------
Ich bestätige die Richtigkeit der gemachten Angaben und Anlagen durch meine Unterschrift.

\feldpar

\TextField[name=datum]{Datum, Unterschrift}, \rule{5cm}{.4pt}

\vspace{4mm}

%%------------------------------------------------------------------------------
%% Vom StuRa auszufüllen
%%------------------------------------------------------------------------------
\begin{tiny}Vom StuRa auszufüllen\end{tiny}

\vspace{-7mm}
\rule{\textwidth}{0.1pt}

%%------------------------------------------------------------------------------
%% Anlagen
%%------------------------------------------------------------------------------
\vspace{-6mm}
\begin{InfoBlock}{Anlagen}

	\CheckBox[name=immatrikulationsbescheinigung]{\parbox{\dimexpr0.5\textwidth-24pt}{Immatrikulationsbescheinigung}}%
	\CheckBox[name=kopie]{\parbox{\dimexpr0.5\textwidth-24pt}{Kopie Studentenausweis}}%
		\\
		\CheckBox[name=bafoeg]{\parbox{\dimexpr0.5\textwidth-24pt}{BAföG-Bescheid bzw. Ablehnungsbescheid}}%
		\CheckBox[name=darstellung]{\parbox{\dimexpr0.5\textwidth-24pt}{Schriftliche Darstellung der Notlage}}%
		\\
		\CheckBox[name=einkommensnachweis]{\parbox{\dimexpr0.5\textwidth-24pt}{Einkommensnachweis}}%
		\CheckBox[name=krankenschein]{\parbox{\dimexpr0.5\textwidth-24pt}{ggf. Krankenschein}}%

\end{InfoBlock}

\vspace{-3mm}
\begin{InfoBlock}{Der Antrag ist}

	\CheckBox[name=angenommen]{\parbox[b]{\dimexpr0.5\textwidth-24pt}{angenommen}}%
	\CheckBox[name=abgelehnt]{\parbox{\dimexpr0.5\textwidth-24pt}{abgelehnt}}%

\end{InfoBlock}

\vspace{-3mm}
\begin{InfoBlock}{Stellungnahme Härtefallausschuss}
	%\parbox[10cm]{\textwidth}{~}
	\begin{minipage}[height=4.5cm]{\textwidth}
		\vspace{4.5cm}
	\end{minipage}
\end{InfoBlock}

\vspace{-3mm}
\begin{InfoBlock}{Angaben zur Rückerstattung}
	\partlinefield[name=rückerstattung]{.1}{.21}{Höhe der Rückerstattung}\euro{}%
\end{InfoBlock}
\end{Form}

\newpage

%%==============================================================================
%% R U E C K S E I T E =========================================================
%%==============================================================================
%\clearpage
	\parbox{\dimexpr\textwidth-3cm}{%
		\noindent%
	\textbf{\huge Antrag auf Härtefallanerkennung}}%
\hfill
\parbox{3cm}{%
%	\IfFileExists{\Logo}{	\includegraphics[width=3cm]{\Logo}	}{	Die Datei \enquote{\Logo} fehlt!	}%
}

\vspace{-8.0mm}
\begin{InfoBlock}{Genehmigung und Anweisung}
	\parbox{.39\textwidth}{Unterschrift Mitglied Härtefallausschuss}\TextField[name=datum1]{Dresden, den} \rule{5cm}{.4pt}\\%
	\parbox{.39\textwidth}{Unterschrift Mitglied Härtefallausschuss}\TextField[name=datum2]{Dresden, den} \rule{5cm}{.4pt}\\%
	\parbox{.39\textwidth}{Unterschrift Mitglied Härtefallausschuss}\TextField[name=datum3]{Dresden, den} \rule{5cm}{.4pt}\\%
	\parbox{.39\textwidth}{Unterschrift Mitglied Referat Finanzen}\TextField[name=datum4]{Dresden, den} \rule{5cm}{.4pt}%
\end{InfoBlock}

%%------------------------------------------------------------------------------
%% Fuss
%%------------------------------------------------------------------------------
\vfill
\noindent
\parbox{.3\linewidth}{
Kontakt:\\
StuRa HTW Dresden\\
Friedrich-List-Platz~1\\
01069~Dresen\\
\hspace{0em}
}
\hfill
\parbox{.3\linewidth}{
\hspace{0em}\\
\href{http://www.stura.htw-dresden.de}{www.stura.htw-dresden.de}\\
\href{mailto:stura@stura.htw-dresden.de}{stura@stura.htw-dresden.de}\\
Telefon: 0351~462-3249\\
Telefax: 0351~462-3240\\
\hspace{0em}
}
\hfill
\parbox{.3\linewidth}{
Bankverbindung:\\
Kontoinhaber: StuRa HTW Dresden\\
IBAN: DE\,098\,505\,030\,031\,201\,115\,45\\
BIC: OSD\,DDE\,81X\,XX\\
Ostsächsische Sparkasse Dresden
}
\end{document}
