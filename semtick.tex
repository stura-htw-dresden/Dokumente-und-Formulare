%%%%%%%%%%%%%%%%%%%%%%%%%%%%%%%%%%%%%%%%%%%%%%%%%%%%%%%%%%%%%%%%%%%%%%%%%%%%%%%%
%% filename:	semtick.tex
%% template:	Mon, 07 May 2012 13:01:14 +0200
%% author:	Hermann Lorenz, geändert von Fabian Kunde, Kleinigkeiten von Wolf, PT 
%% date:
%%	07. Jun 2017	KaZ0
%%		* Antragskopf überarbeitet
%%		* studentisches Jahresticket ersetzt durch Semesterticket
%%		* Nachkauf/Wiedernutzung ersetzt durch Nachkauf
%%		* SS und WS ersetzt durch SoSe und WiSe
%%		* Aus Abschnitt Grund CheckBox Nachkauf/Wiedernutzung entfernt
%%		* Anpassung Beiträge für Befreiung, Rückerstattung und Nachkauf
%%		* Format Beitragstabelle angepasst
%%	23. Aug 2016 	joshiee
%%		* insert 29,30 (1/6) bei Betrag für Nachkauf
%%	28. Jun 2016 	PT
%%		* Nachkauf zu Nachkauf/Wiedernutzung
%%		* "die Chipkarte" zu "der Studentenausweis"
%%		* Streichungen von "Chipkarte" und "Nur möglich, wenn
%%			vorher keine Rückerstattung, sondern eine Befreiung erfolgt ist."
%%		* Reduziere StuRa-/StudSek-Box von 14.8em auff 14.2em
%%	21. Nov 2015	Wolf (auf der Suche nach Beispielen für PDF-Formulare)
%%		* Logo wirft keine Fehlermeldung mehr wenn die externe Datei fehlt
%%		* Fallback ist das aktuelle SturaLogo und fest hier eingebettet
%%	25. Jun 2015	Fabian Kunde
%%		* Beträge für das Jahresticket aktualisiert
%%		* Matrikelnummer durch s-Nummer ersetzt
%%		* Neues Logo integriert (Achtung:Datei wurde angepasst!!!! Dateiname muss passen oder ggf. angepasst werden)
%%		* Datum wird automatisch auf das aktuelle gestzt
%%	23. Jan 2014	Hermann Lorenz
%%		* alle Felder mit symbolischen Namen versehen (vorher wurden
%%		  mehrere Felder unter AdobeReader immer mit dem selben Inhalt
%%		  versehen)
%%		* dazu wurden auch die neuen Befehle \partlinefield und
%%		* \halflinefield angepasst
%%		* Rechtschreibfehler korrigiert
%%		* Quelltextformatierungen auf 80 Zeichen begrenzt
%%	16. Jan 2014 Fabian Kunde
%%		* ???
%%	21. Jan 2014 Fabian Kunde
%%		* ???
%%	24. Jun 2013 14:18  Hermann Lorenz
%%		* erstellt
%% Übersetzen:
%%	`mkdir -p aux`
%%	`lualatex -output-directory aux semtick.tex`
%%	Dazu soll die Datei 'stura-logo.png' (mit dem aktuellen Logo) im Verzeichnis liegen.
%%	Der Nahme der Datei kann selbstverständlich angepasst werden, er muss nur im Feld
%%	\includegraphics[<Ausrichtung Text in der Box>(t,c,b)][<Breite>(width=) und/oder <Höhe>(height=)]{<Dateiname>}
%%	eingetragen werden.
%%%%%%%%%%%%%%%%%%%%%%%%%%%%%%%%%%%%%%%%%%%%%%%%%%%%%%%%%%%%%%%%%%%%%%%%%%%%%%%%
\documentclass[%
	ngerman,	% an alle Packete weitergeben
	parskip=half,
	% für Package typearea
	paper=a4,% Druckbereich A4
	pagesize=auto	% Druckbereich als Papiergröße über-
			% nehmen
	]{scrartcl}


%%%%%%%%%%%%%%%%%%%%%%%%%%%%%%%%%%%%%%%%%%%%%%%%%%%%%%%%%%%%%%%%%%%%%%%%%%%%%%%%
%% LuaLaTeX gesondert behandeln %%%%%%%%%%%%%%%%%%%%%%%%%%%%%%%%%%%%%%%%%%%%%%%%
%%%%%%%%%%%%%%%%%%%%%%%%%%%%%%%%%%%%%%%%%%%%%%%%%%%%%%%%%%%%%%%%%%%%%%%%%%%%%%%%
\usepackage{ifluatex}

\usepackage{geometry}
\usepackage{scrpage2}
\geometry{left=1cm,right=1cm,top=1cm,bottom=1.5cm}

%%%%%%%%%%%%%%%%%%%%%%%%%%%%%%%%%%%%%%%%%%%%%%%%%%%%%%%%%%%%%%%%%%%%%%%%%%%%%%%%
%% Lokalisierung %%%%%%%%%%%%%%%%%%%%%%%%%%%%%%%%%%%%%%%%%%%%%%%%%%%%%%%%%%%%%%%
%%%%%%%%%%%%%%%%%%%%%%%%%%%%%%%%%%%%%%%%%%%%%%%%%%%%%%%%%%%%%%%%%%%%%%%%%%%%%%%%
\ifluatex
	\usepackage{fontspec}
\else
	\usepackage[utf8]{inputenx}	% Umlaute direkt eingeben
	\usepackage[T1]{fontenc}	% Wörter mit Umlaute umbrechen
\fi

\usepackage{babel}				% deutsche Bezeichner
\usepackage[autostyle,german=guillemets]{csquotes}	% \enquote{}


%%%%%%%%%%%%%%%%%%%%%%%%%%%%%%%%%%%%%%%%%%%%%%%%%%%%%%%%%%%%%%%%%%%%%%%%%%%%%%%%
%% Font %%%%%%%%%%%%%%%%%%%%%%%%%%%%%%%%%%%%%%%%%%%%%%%%%%%%%%%%%%%%%%%%%%%%%%%%
%%%%%%%%%%%%%%%%%%%%%%%%%%%%%%%%%%%%%%%%%%%%%%%%%%%%%%%%%%%%%%%%%%%%%%%%%%%%%%%%
\ifluatex
	\setmainfont[Mapping=tex-text]{Linux Libertine O}
	\setsansfont[Mapping=tex-text]{Linux Biolinum O}
	\setmonofont[Mapping=tex-text,Scale=0.85]{DejaVu Sans Mono}
\else
	\usepackage{libertine}
	\usepackage[scaled=0.85]{DejaVuSansMono}
\fi

%%%%%%%%%%%%%%%%%%%%%%%%%%%%%%%%%%%%%%%%%%%%%%%%%%%%%%%%%%%%%%%%%%%%%%%%%%%%%%%%
%% Tabellen %%%%%%%%%%%%%%%%%%%%%%%%%%%%%%%%%%%%%%%%%%%%%%%%%%%%%%%%%%%%%%%%%%%%
%%%%%%%%%%%%%%%%%%%%%%%%%%%%%%%%%%%%%%%%%%%%%%%%%%%%%%%%%%%%%%%%%%%%%%%%%%%%%%%%
\usepackage{tabularx}
\usepackage{booktabs}	% \toprule\midrule\bottomrule
			% \addlinespace

%%%%%%%%%%%%%%%%%%%%%%%%%%%%%%%%%%%%%%%%%%%%%%%%%%%%%%%%%%%%%%%%%%%%%%%%%%%%%%%%
%% Bilder %%%%%%%%%%%%%%%%%%%%%%%%%%%%%%%%%%%%%%%%%%%%%%%%%%%%%%%%%%%%%%%%%%%%%%
%%%%%%%%%%%%%%%%%%%%%%%%%%%%%%%%%%%%%%%%%%%%%%%%%%%%%%%%%%%%%%%%%%%%%%%%%%%%%%%%
\usepackage{graphicx}	% \includegraphics{bild.pdf}
\usepackage{tikz}
\usepackage{environ}

%%%%%%%%%%%%%%%%%%%%%%%%%%%%%%%%%%%%%%%%%%%%%%%%%%%%%%%%%%%%%%%%%%%%%%%%%%%%%%%%
%% Mathematische Symbole %%%%%%%%%%%%%%%%%%%%%%%%%%%%%%%%%%%%%%%%%%%%%%%%%%%%%%%
%%%%%%%%%%%%%%%%%%%%%%%%%%%%%%%%%%%%%%%%%%%%%%%%%%%%%%%%%%%%%%%%%%%%%%%%%%%%%%%%
\usepackage{amssymb}
\usepackage{amsmath}
\ifluatex
	\usepackage{unicode-math}
	\setmathfont{Asana-Math.otf}
	%\setmathfont{lmmath-regular.otf}
	%\setmathfont{xits-math.otf}
\else
	\usepackage{amssymb}
	\usepackage{amsfonts}
\fi
\usepackage{nicefrac}

%%%%%%%%%%%%%%%%%%%%%%%%%%%%%%%%%%%%%%%%%%%%%%%%%%%%%%%%%%%%%%%%%%%%%%%%%%%%%%%%
%% Sonstige Symbole %%%%%%%%%%%%%%%%%%%%%%%%%%%%%%%%%%%%%%%%%%%%%%%%%%%%%%%%%%%%
%%%%%%%%%%%%%%%%%%%%%%%%%%%%%%%%%%%%%%%%%%%%%%%%%%%%%%%%%%%%%%%%%%%%%%%%%%%%%%%%
\usepackage{xspace}
\usepackage{paralist}
\usepackage{xcolor}

%%%%%%%%%%%%%%%%%%%%%%%%%%%%%%%%%%%%%%%%%%%%%%%%%%%%%%%%%%%%%%%%%%%%%%%%%%%%%%%%
%% pdf-links %%%%%%%%%%%%%%%%%%%%%%%%%%%%%%%%%%%%%%%%%%%%%%%%%%%%%%%%%%%%%%%%%%%
%%%%%%%%%%%%%%%%%%%%%%%%%%%%%%%%%%%%%%%%%%%%%%%%%%%%%%%%%%%%%%%%%%%%%%%%%%%%%%%%
\usepackage{hyperref} %macht auch for Formulare
\definecolor{linkcolor}{rgb}{.2,.2,.7}
\hypersetup{
	colorlinks,	% farbige Links, nicht umrandet
	linkcolor=linkcolor,	% Farbe fuer dokumentinterne Links
	urlcolor=linkcolor,	% Farbe fuer dokumentexterne Links
	linktoc=all,	% in Verzeichnissen Zahlen und Texte verlinken
	unicode,	% Umlaute in PDF-Strings
	}

%%%%%%%%%%%%%%%%%%%%%%%%%%%%%%%%%%%%%%%%%%%%%%%%%%%%%%%%%%%%%%%%%%%%%%%%%%%%%%%%
%% eigene Macros %%%%%%%%%%%%%%%%%%%%%%%%%%%%%%%%%%%%%%%%%%%%%%%%%%%%%%%%%%%%%%%
%%%%%%%%%%%%%%%%%%%%%%%%%%%%%%%%%%%%%%%%%%%%%%%%%%%%%%%%%%%%%%%%%%%%%%%%%%%%%%%%
\newcommand{\nachweis}[1]{\emph{(#1)}}

\newsavebox{\InfoBlockBox}
\newenvironment{InfoBlock}[1]{%
	\par\vspace{1em}\noindent\textbf{\textsf{\large#1}}\par\noindent%
		\begin{lrbox}{\InfoBlockBox}%
		\begin{minipage}{\dimexpr\linewidth-2\fboxrule-2\fboxsep}%
	}{%
		\end{minipage}%
		\end{lrbox}
		\fbox{\usebox{\InfoBlockBox}}%
	}

%%==============================================================================
%% Allgemeine Formatierungen/Einstellungen
%%==============================================================================
\definecolor{bg-gray}{gray}{.9}
\linespread{1.25}	% 1.5-facher Zeilenabstand (1.2 * 1.25 = 1.5)

\ofoot{Stand: \today\\[1cm]}
\cfoot{\thepage\\[1cm]}
\pagestyle{scrheadings}


%%==============================================================================
%% Formatierungen der Formularfelder
%%==============================================================================
\def\DefaultHeightofCheckBox{8px}
\def\DefaultWidthofCheckBox{8px}
\def\DefaultHeightofTextField{12px}

\def\LayoutCheckField#1#2{\mbox{\mbox{#2}~#1}\xspace}
\def\LayoutTextField#1#2{\mbox{\mbox{#1}~~\mbox{#2}}\xspace}


%%==============================================================================
%% Formatierungsbefehle
%%==============================================================================
\newcommand{\feldumbruch}{\\[1ex]}
\newcommand{\feldpar}{\vspace{\dimexpr1ex+2ex}}
\newcommand{\halflinefield}[2][]{%
	\partlinefield[#1]{.32}{.15}{#2}%
	}
\newcommand{\partlinefield}[4][]{%
	\makebox[\dimexpr#2\linewidth+#3\linewidth+1em][l]{%
		\TextField[width=#2\linewidth,#1]
			{\makebox[#3\linewidth][r]{#4}}%
		}%
	}


%%==============================================================================
%% PDF-Metadaten
%%==============================================================================
\hypersetup{
	pdftitle={Antrag auf Rückerstattung/Befreiung/Nachkauf},
	pdfauthor={StuRa HTW Dresden},
	}

%\newcommand{\StuRaLogoFile}{./stura-logo.png}%jpg und gif gingen auch, dann iss aber genauso (pixel-)kacke
\newcommand{\StuRaLogoFile}{./stura-logo.pdf}%nehmt bitte eine saubere SVG (z.B. bei https://github.com/stura-htw-dresden/htw-logo/), Anleitung PDF bauen ist dabei
\IfFileExists{\StuRaLogoFile}{}{
	\definecolor{schwarz}{RGB}{0,0,0}
	\definecolor{htworange}{RGB}{240,172,43}
}

%define scaling for tikz
\makeatletter
\newsavebox{\measure@tikzpicture}
\NewEnviron{scaletikzpicturetowidth}[1]{%
  \def\tikz@width{#1}%
  \def\tikzscale{1}\begin{lrbox}{\measure@tikzpicture}%
  \BODY
  \end{lrbox}%
  \pgfmathparse{#1/\wd\measure@tikzpicture}%
  \edef\tikzscale{\pgfmathresult}%
  \BODY
}
\NewEnviron{scaletikzpicturetoheight}[1]{%
  \def\tikz@width{#1}%
  \def\tikzscale{1}\begin{lrbox}{\measure@tikzpicture}%
  \BODY
  \end{lrbox}%
  \pgfmathparse{#1/\ht\measure@tikzpicture}%
  \edef\tikzscale{\pgfmathresult}%
  \BODY
}
\makeatother

\begin{document}
\begin{Form}

%%------------------------------------------------------------------------------
%% Antragskopf
%%------------------------------------------------------------------------------
	\parbox{\dimexpr\linewidth-4,5cm}{\LARGE\textsf Antrag zum Semesterticket auf~}
\hfill
%%------------------------------------------------------------------------------
%% Logo
%%------------------------------------------------------------------------------
\IfFileExists{\StuRaLogoFile}{
	\newcommand{\StuRaLogo}{%
		\includegraphics*[width=4cm]{\StuRaLogoFile}%
	}
}{
	% nothing, inline or input the Logo as you wish
	\newcommand{\StuRaLogo}{\begin{tikzpicture}[y=0.80pt, x=0.80pt, yscale=-0.15, xscale=0.15, inner sep=0pt, outer sep=0pt]%[scale=\tikzscale]
	\path[fill=schwarz] (131.9062,0.0000) .. controls (121.3063,0.0000) and
	  (110.8063,4.9937) .. (100.4062,15.0938) .. controls (98.9062,16.7937) and
	  (93.2875,23.3938) .. (92.1875,25.0938) .. controls (83.0359,40.5922) and
	  (72.7150,56.9183) .. (64.6875,70.4062) .. controls (63.5875,72.0062) and
	  (61.0063,78.7062) .. (60.4062,80.4062) -- (60.3125,80.4062) .. controls
	  (58.5125,90.5062) and (62.7875,95.5005) .. (73.1875,95.5000) --
	  (157.5938,95.5000) -- (130.0625,140.8125) -- (21.5938,140.8125) --
	  (0.0000,176.0000) -- (120.5000,176.0000) .. controls (131.2000,176.0000) and
	  (141.6000,171.0062) .. (152.0000,160.9062) -- (151.9063,160.9062) .. controls
	  (153.3063,159.2063) and (159.0063,152.5125) .. (159.9063,150.8125) .. controls
	  (173.8596,128.6079) and (188.1732,104.8225) .. (200.0000,85.4062) .. controls
	  (201.1000,83.8063) and (201.9938,82.1063) .. (202.5938,80.4062) .. controls
	  (204.4938,70.4062) and (201.8875,60.3125) .. (191.1875,60.3125) --
	  (106.9063,60.3125) -- (122.4063,35.1875) -- (230.9063,35.1875) --
	  (243.1875,15.0938) -- (252.5000,0.0000) -- cycle(266.9062,0.0000) .. controls
	  (259.7267,11.7404) and (252.5209,23.4648) .. (245.3125,35.1875) --
	  (301.8750,35.1875) -- (215.6875,176.0000) -- (251.9062,176.0000) .. controls
	  (280.8223,129.2741) and (309.3841,82.1635) .. (338.0625,35.1875) --
	  (390.0000,35.1875) -- (411.5938,0.0000) -- cycle(426.0938,0.0000) --
	  (339.8125,140.8125) -- (333.5938,150.8125) .. controls (332.5938,152.5125) and
	  (330.3000,159.2063) .. (329.5000,160.9062) .. controls (327.5000,171.0062) and
	  (331.8062,176.0000) .. (342.4062,176.0000) -- (438.9062,176.0000) .. controls
	  (449.5063,176.0000) and (460.0125,171.0062) .. (470.3125,160.9062) .. controls
	  (471.7125,159.2063) and (477.4125,152.5125) .. (478.3125,150.8125) --
	  (484.5000,140.8125) .. controls (513.2206,93.8698) and (541.9568,46.9366) ..
	  (570.6875,0.0000) -- (534.5938,0.0000) -- (448.3125,140.8125) --
	  (376.0000,140.8125) -- (462.1875,0.0000) -- cycle(804.5938,0.0000) --
	  (629.3125,176.0938) -- (670.3125,176.0938) -- (710.3125,135.8125) --
	  (778.3125,135.8125) -- (769.0937,176.0938) -- (810.0937,176.0938) --
	  (850.4062,0.0000) -- cycle(798.6875,47.0000) -- (786.3125,100.5938) --
	  (745.3125,100.5938) -- cycle;
	\path[fill=htworange] (585.1875,0.0000) -- (477.3125,176.0000) --
	  (513.5000,176.0000) .. controls (530.0935,149.3168) and (546.4228,122.4480) ..
	  (562.8438,95.5938) -- (633.6250,95.5938) -- (585.8125,176.0000) --
	  (622.0000,176.0000) -- (665.1875,105.5938) .. controls (666.2875,103.9938) and
	  (668.7125,97.2938) .. (669.3125,95.5938) -- (669.3125,95.5000) .. controls
	  (671.2791,85.4308) and (668.7553,75.4062) .. (658.1875,75.4062) .. controls
	  (677.8048,76.3801) and (683.9009,66.8949) .. (690.8125,60.3125) .. controls
	  (692.0125,58.7125) and (697.9000,52.0125) .. (699.0000,50.3125) --
	  (714.4062,25.0938) .. controls (715.4062,23.3938) and (716.3937,21.7937) ..
	  (717.0938,20.0938) .. controls (723.0938,4.2937) and (716.3875,0.0000) ..
	  (705.6875,0.0000) -- cycle(599.8125,35.1875) -- (672.0938,35.1875) --
	  (656.5000,60.3125) -- (584.1875,60.3125) -- cycle;
	\end{tikzpicture}}%
}
%important to prevent linebreak
\parbox[3cm]{4cm}{%
%	\begin{scaletikzpicturetowidth}{\textwidth}
		\StuRaLogo%
%	\end{scaletikzpicturetowidth}
	}\\[2ex]
%%------------------------------------------------------------------------------
%% Antragsart/Semester
%%------------------------------------------------------------------------------
	\parbox{.5\linewidth}{%
	\noindent
{\CheckBox[name=antragbefreiung]{\LARGE\textsf{Befreiung,}}}}%
\hfill
\parbox{.5\linewidth}{%
	\noindent
	{\CheckBox[name=antragrueckerstattung]{\LARGE\textsf{Rückerstattung}}%
		\,\LARGE\textsf{oder}\,
\CheckBox[name=antragnachkauf]{\LARGE\textsf{Nachkauf}}}}
\parbox{.38\linewidth}{~}
\hfill
\parbox{.62\linewidth}{%
%\vspace{-2ex}
	%\center{für das}
	ab dem
}\\[2ex]
\parbox[b]{.5\linewidth}{%
	\noindent
	\CheckBox[name=fuerss]{\makebox[2.5cm][l]{Sommersemester}}
	\TextField[width=2cm,name=ssjahr1]{~}\,(Jahr)
}%
\hfill%
\parbox[b]{.5\linewidth}{%
	\noindent%
	\parbox{3.08cm}{~}\TextField[width=2cm,name=wsjahr1]{~}\,(Monat)\feldumbruch%
		\CheckBox[name=fuerws]{\makebox[2.5cm][l]{Wintersemester}}
		\TextField[width=2cm,name=wsjahr1]{~}~~/~%
		\TextField[width=2cm,name=wsjahr2]{~}\,(Jahr)
}%

%%------------------------------------------------------------------------------
%% allgemeine Angaben
%%------------------------------------------------------------------------------
\begin{InfoBlock}{allgemeine Angaben}
	\partlinefield[name=nachname]{.32}{.15}{Name}%
	\partlinefield[name=vorname]{.37}{.10}{Vorname}%
	\feldumbruch
	\partlinefield[name=bibnr]{.15}{.15}{Bibl.-Nr.}

	\feldpar

	\partlinefield[name=strasse]{.5}{.15}{Straße}\feldumbruch
	\partlinefield[name=plz]{.1}{.15}{PLZ, Ort},%
		\TextField[width=.5\linewidth,name=ort]{~}

	\feldpar

	\partlinefield[name=telefon]{.32}{.15}{Telefonnummer}%
	\partlinefield[name=email]{.37}{.10}{E-Mail}%
\end{InfoBlock}

%%------------------------------------------------------------------------------
%% Bankverbindung
%%------------------------------------------------------------------------------
\begin{InfoBlock}{Bankverbindung (nur bei Rückerstattung)}
	\partlinefield[name=iban]{.37}{.15}{IBAN}%
	\partlinefield[name=bic]{.32}{.10}{BIC}%
	\feldumbruch
	\halflinefield[name=kontoinhaber]{Kontoinhaber}%
	\halflinefield[name=kreditinstitut]{Kreditinstitut}%
\end{InfoBlock}

%%------------------------------------------------------------------------------
%% Grund
%%------------------------------------------------------------------------------
\begin{InfoBlock}{Befreiungs-/Erstattungsgrund (Nachweise in Klammern)}
	\CheckBox[name=begruendungausland]{Auslandssemester ohne Beurlaubung}
		\nachweis{Nachweis der Hochschule}\\
	\CheckBox[name=begruedungpraktikum]{Praktikumssemester außerhalb des
		Gültigkeitsbereiches des Tickets im VVO}
		\nachweis{Praktikumsvereinbarung}\\
	\CheckBox[name=begruendungabschlussarbeit]{Abschlussarbeit außerhalb des
		Gültigkeitsbereiches des Tickets im VVO}
		\nachweis{entsprechende Vereinbarung}\\
	\CheckBox[name=begruendungexma]{Exmatrikulation im laufenden Semester}
		\nachweis{Exmatrikulationsbescheinigung}\\
	\CheckBox[name=begruedungurlaub]{Urlaubssemester}
		\nachweis{genehmigter Urlaubssemesterantrag oder
		Immatrikulationsbescheinigung mit dem Vermerk
		Urlaubssemester}\feldumbruch
	\CheckBox[name=begruendungsonstiges]{Sonstiges}
		\TextField[width=.8\linewidth,name=begruedungsonstigestext]{~}
\end{InfoBlock}

%%------------------------------------------------------------------------------
%% Unterschrift
%%------------------------------------------------------------------------------
Mit meiner Unterschrift bestätige ich, dass ich die Beitragsordnung zur
Kenntnis genommen habe.

\feldpar

\TextField[name=datum]{Datum, Unterschrift}, \rule{5cm}{.4pt}


%%------------------------------------------------------------------------------
%% StuRa-/StudSek-Box
%%------------------------------------------------------------------------------
	\vfill

\noindent%
\frame{\colorbox{bg-gray}
{%
	\parbox[top][14.2em][t]{\dimexpr.5\linewidth-2\fboxrule-7mm}
	{%
		\begin{tiny}Vom StuRa auszufüllen\end{tiny}\\[2ex]
		\makebox[2.5cm][r]{Antragsdatum}~\rule{5cm}{.4pt}\\[2ex]
		\makebox[2.5cm][r]{Betrag}~\rule{3cm}{.4pt}\\[2ex]
		\begin{small}Signum/Stempel:\end{small}
	}%
}}%
\hfill%
\frame{\colorbox{bg-gray}
{%
	\parbox[top][14.2em][t]{\dimexpr.5\linewidth-2\fboxrule-7mm}
	{%
		\begin{tiny}Vom StudSek auszufüllen\end{tiny}\\[2ex]
		\makebox[2.5cm][r]{Datum}~\rule{5cm}{.4pt}\\[2ex]
		\\[2ex]
		\begin{small}Signum/Stempel:\end{small}
	}%
}}
\end{Form}



%%==============================================================================
%% R U E C K S E I T E =========================================================
%%==============================================================================
\clearpage
\linespread{1}	% 1.2-facher Zeilenabstand (1.2 * 1.0 = 1.2)

%%------------------------------------------------------------------------------
%% Firmennachweis
%%------------------------------------------------------------------------------
\noindent
Nachweis zum Antrag:

\noindent\hspace*{2em}\partlinefield[name=firma]{.5}{.15}{Firma}\feldumbruch
\hspace*{2em}\partlinefield[name=firmenanschrift]{.5}{.15}{Anschrift}%
	\feldumbruch
\hspace*{2em}\partlinefield[name=firmenanstellung]{.5}{.15}{Zeitraum der
	Anstellung}\feldumbruch
\hspace*{2em}\makebox[.15\linewidth][r]{Signum/Stempel}

%\vspace{2.2cm}
\vfill

%%------------------------------------------------------------------------------
%% Hinweistexte
%%------------------------------------------------------------------------------
\hrule
Liebe Studentinnen und Studenten,\\
dieses Formblatt bietet Euch drei Möglichkeiten einen Antrag zu stellen:
\footnotesize
%\small{
\begin{compactenum}
\item	Antrag auf Befreiung für ein Semester vom Semesterticket
	(Nur möglich, vor Ablauf des Rückmeldezeitraums und wenn der
	Semesterbeitrag noch nicht bezahlt wurde.)

\item	Antrag auf Rückerstattung des Semestertickets für ein
	Semester (Nur möglich, wenn der Semesterbeitrag vollständig gezahlt
	wurde und der Studentenausweis bei Antragstellung vorliegt.)

\item	Antrag auf Nachkauf des Semestertickets (Die
	Nachzahlung der Beitragsrate für das Semestersticket kann per
	Überweisung an den oder per Barzahlung im StuRa erfolgen. Eine
	Kartenzahlung ist nicht möglich!)
\end{compactenum}

Eine Befreiung/Rückerstattung kann nur erfolgen, wenn sich der/die Studierende
aufgrund der auf der Vorderseite aufgeführten Gründe außerhalb des
Verkehrsverbundes Oberelbe (VVO) aufhält oder exmatrikuliert ist.

Zur Genehmigung muss der entsprechende Nachweis angefügt werden (Kopie der
entsprechenden Vereinbarung bzw. der Imma- oder Exmatrikulationsbescheinigung).
Es werden nur vollständige Anträge bearbeitet!

\minisec{Abgabefristen:}
\begin{compactitem}
\item[\emph{Befreiung:}]
	Bis zum Ende des Rückmeldezeitraums.
\item[\emph{Rückerstattung:}]
	Entsprechend der unten aufgeführten Tabelle erfolgt die Erstattung
	anteilmäßig. Für jeden angefangenen und verstrichenen Monat der
	Gültigkeit wird $\nicefrac{1}{6}$ weniger vom Beitrag für das
	studentische Ticket des jeweiligen Semesters erstattet.
\end{compactitem}

\begin{center}
\begin{tabular}{cccccc}
	\toprule
		WiSe bis \dots
		& SoSe bis \dots
		& Betrag für Rückerstattung in Euro
		& WiSe bis \dots
		& SoSe bis \dots
		& Betrag für Nachkauf in Euro
		\tabularnewline
	\midrule
			31.\,08.
			& 28. bzw. 29.\,02.
			& 184,80 ($\nicefrac{6}{6}$)
			& 31.\,08.
			& 28. bzw. 29.\,02.
			& 184,80 ($\nicefrac{6}{6}$)
			\tabularnewline
			30.\,09.
			& 31.\,03.
			& 154,00 ($\nicefrac{5}{6}$)
			& 30.\,09.
			& 31.\,03.
			& 184,80 ($\nicefrac{6}{6}$)
			\tabularnewline
			31.\,10.
			& 30.\,04.
			& 123,20 ($\nicefrac{4}{6}$)
			& 31.\,10.
			& 30.\,04.
			& 154,00 ($\nicefrac{5}{6}$)
			\tabularnewline
			30.\,11.
			& 31.\,05.
			& ~~92,40 ($\nicefrac{3}{6}$)
			& 30.\,11.
			& 31.\,05.
			& 123,20 ($\nicefrac{4}{6}$)
			\tabularnewline
			31.\,12.
			& 15.\,06.
			& ~~61,60 ($\nicefrac{2}{6}$)
			& 31.\,12.
			& 30.\,06.
			& ~~92,40 ($\nicefrac{3}{6}$)
			\tabularnewline
			15.\,01.
			& --- 
			& ~~30,80 ($\nicefrac{1}{6}$)
			& 31.\,01.		
			& 31.\,07.
			& ~~61,60 ($\nicefrac{2}{6}$)
			\tabularnewline
			\multicolumn{2}{c}{danach}
			& keine Rückerstattung
			&	\multicolumn{2}{c}{danach}
			& ~~30,80 ($\nicefrac{1}{6}$)
			\tabularnewline
	\bottomrule
\end{tabular}
\end{center}

\emph{Aus technischen Gründen kann eine Entwertung der Studentenausweise
und damit eine Rückerstattung nur bis spätestens zum
15.\,06.~(SoSe)/""15.\,01.~(WiSe) erfolgen. Die Anträge sind rechtzeitig zu stellen!}

Bei Rückerstattung muss der Semesterbeitrag bereits vollständig eingezahlt
worden sein und der Studentenausweis muss bei Antragsabgabe
vorliegen, damit dieser vom Studentensekretariat zurückgesetzt werden kann.

Die Überweisung kann aus verwaltungstechnischen Gründen mehrere Wochen
dauern. Auf eine Bearbeitung des Antrages innerhalb der vorlesungsfreien
Zeit besteht kein Anspruch!

Grundlage für die Befreiung bzw. Rückerstattung von studentischen
Jahresticketbeiträgen und dem Nachkauf des Semestertickets ist die
Beitragsordnung der Studentinnen- und Studentenschaft der Hochschule für Technik
und Wirtschaft Dresden sowie die Verträge zwischen den Studentenräten der
Hochschulen in der Stadt Dresden und dem VVO sowie der DVB AG und der DB Regio
AG über das Semesterticket.

Bitte informiere Dich auch auf der Internetseite des StuRa:
\url{http://www.stura.htw-dresden.de}

%%------------------------------------------------------------------------------
%% Fuss
%%------------------------------------------------------------------------------
\noindent
\parbox{.3\linewidth}{
Anschrift:\\
StuRa HTW Dresden\\
Friedrich-List-Platz~1\\
01069~Dresden\\
\hspace{0em}
}
\hfill
\parbox{.3\linewidth}{
\hspace{0em}\\
\href{mailto:ticket@stura.htw-dresden.de}{ticket@stura.htw-dresden.de}\\
Telefon: 0351~462-3249\\
Telefax: 0351~462-3240\\
\hspace{0em}
}
\hfill
\parbox{.3\linewidth}{
Bankverbindung:\\
Kontoinhaber: StuRa HTW Dresden\\
IBAN: DE\,098\,505\,030\,031\,201\,115\,45\\
BIC: OSD\,DDE\,81X\,XX\\
Ostsächsische Sparkasse Dresden
}
%}
\end{document}
